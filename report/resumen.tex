\chapter{Resumen}

El análisis de la marcha es un estudio clínico sencillo que colaborar en el diagnóstico y control de pacientes con alguna forma de dificultad motora. El Laboratorio de Investigación en Reconocimiento de Patrones y Sistemas Inteligentes (PRIS-Lab) de la Escuela de Ingeniería Eléctrica de la Universidad de Costa Rica cuenta con equipo de captura de movimiento apropiado para el estudio del movimiento humano. Desde este Laboratorio se propone la creación de un sistema computacional que permita a profesionales de diversas disciplinas colaborar en la investigación del movimiento humano, desempeño deportivo y salud. Este trabajo representa un primer acercamiento a este sistema computacional, se limitó el desarrollo a los estudios de la marcha, tema que ya se ha tratado en el laboratorio. A través de una búsqueda bibliográfica se determinaron las variables cinemáticas y espacio-temporales, así como las técnicas de análisis usadas con mayor frecuencia en la investigación, a partir de lo encontrado se desarrollo un marco de trabajo en \emph{software} que permitiría a los investigadores comprobar hipótesis rápidamente y desarrollar nuevas técnicas de análisis y relaciones entre variables cinéticas y espacio-temporales con las afectaciones motoras. 
