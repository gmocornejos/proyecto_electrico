\chapter{Introducción}

\comentario{Voy a dejar esta sección para el final} \\

Nota: Habar (en ese orden) 

Justificación: Por qué el análisis de la marcha y su estudio es importante

Problema: Cuál es el problema puntual que se va a resolver 

Alcances: Qué cosas se hicieron y qué quedó por fuera. 


\section{Objetivos}

\subsection{Objetivo general}

Desarrollar un \emph{software framework} para facilitar a investigadores de distintas disciplinas analizar las variables cinemáticas y espacio-temporales de la marcha, a partir de datos recolectados por un sistema de captura óptica de movimiento.

\subsection{Objetivos específicos}

\begin{enumerate}
    \item Realizar una investigación bibliográfica sobre las variables cinemáticas y espacio temporales de la marcha y técnicas de análisis de la marcha.
    \item Seleccionar las variables cinemáticas y espacio-temporales más relevantes, así como las técnicas más utilizadas al realizar un análisis de la marcha. 
    \item Plantear los requerimientos técnicos y de uso para la solución. 
    \item Implementar un \emph{software framework} capaz de tomar datos de un sistema de captura óptica de movimiento y aplicar técnicas de análisis usuales a variables cinemáticas y espacio-temporales de la marcha, según los requerimientos técnicos y de uso planteados.
    \item Evaluar el \emph{software}, tanto a nivel técnico como de uso.
    \item Elaborar documentación del \emph{software} e informe final del proyecto.
\end{enumerate}

\section{Desarrollo}

\comentario{¿Qué se va a tratar en cada capítulo?}

Los capítulos \ref{generalidades} y \ref{variables-tecnicas} comprenden el marco teórico del trabajo. En el capítulo \ref{generalidades}: \nameref{generalidades} introduce la marcha, su utilidad como herramienta de diagnóstico, hardware y software utilizado en su estudio. El capitulo \ref{variables-tecnicas}: \nameref{variables-tecnicas} hace una revisión bibliográfica para establecer las variables cinemáticas, espacio-temporales y técnicas de análisis más utilizadas en el campo. 
