% Universidad de Costa Rica
% Escuela de Ingeniería Eléctrica
% IE0499: Proyecto Eléctrico. 
% Primer semestre, 2016

% Título: Desarrollo de un software para el análisis de variables cinemáticas y espacio-temprales de la marcha
% Autor: Luis Guillermo Cornejo Suárez luis.cornejo@ucr.ac.cr
% Supervisores: Dr. rer. nat. Francisco Siles Canales francisco.siles@ucr.ac.cr
%               Ing. Andrés Mora Zúñiga amora@eie.ucr.ac.cr
%               Bach. Ana María Gomez Granados

% ==== Preambulo ====

\documentclass[11pt, letterpaper, twoside, openright]{book}

\usepackage[spanish]{babel}
\usepackage[T1]{fontenc}
\usepackage[utf8]{inputenc}
\usepackage{amsmath}
\usepackage{fancyhdr}
\usepackage{graphicx}
\usepackage{url}
\usepackage{booktabs}
\usepackage{float}
\usepackage{SIunits}
\usepackage{geometry}
\usepackage{natbib}
\usepackage{longtable}
\usepackage{caption}

\usepackage{nameref}
\usepackage{color}

\bibliographystyle{apalike_es}
\setcounter{secnumdepth}{1} % Solo numera capítulos y secciones
\newcommand{\newchap}[1]{\include{#1} \clearpage{\pagestyle{empty}\cleardoublepage}}
\newcommand{\comentario}[1]{{\color{red} #1}}
\newcommand{\mono}[1]{{\ttfamily #1}}
\newcommand{\equwref}[1]{(\ref{#1})}
% ==== Inicio del informe ==== 
\begin{document}
\frontmatter
\pagestyle{plain}

% ---- Portada ----
\thispagestyle{empty}
\newgeometry{margin=2cm}
\begin{center}
     \Large\textsc{Universidad de Costa Rica \\ Facultad de Ingeniería \\ Escuela de Ingeniería Eléctrica} \\
    \vspace{6cm}
    \LARGE\bfseries{Desarrollo de un \emph{software} para el análisis de variables cinemáticas y espacio-temporales de la marcha} \\
    \vspace{2cm}
    \large\normalfont{Por:} \\
    \vspace{0.5cm}
    \Large\normalfont{Luis Guillermo Cornejo Suárez} \\
%    \vspace{9cm}
    \vfill
    \large{Ciudad Universitaria Rodrigo Facio Brenes}\\
    \vspace{0.5cm}
    \large{Julio 2016}
\end{center}
\newpage\null\thispagestyle{empty}\newpage

% ---- Aprobación ----
\thispagestyle{empty}
\begin{center}
    \LARGE\bfseries{Desarrollo de un \emph{software} para el análisis de variables cinemáticas y espacio-temporales de la marcha} \\
    \vspace{2cm}
    \large\normalfont{Por:} \\
    \vspace{0.5cm}
    \Large\normalfont{Luis Guillermo Cornejo Suárez} \\
    \vfill
    \large\bfseries{IE0499 - Proyecto Eléctrico} \\
    \vspace{0.5cm}
    \large\normalfont{Aprobado por el Tribunal:} \\
    \vspace{4cm}
    \rule{6cm}{0.1pt}\\
    \normalfont\large{Dr. rer. nat. Francisco Siles Canales} \\
    \normalfont\large{Profesor guía} \\
    \vspace{2cm} 
    \begin{table}[!h]
        \centering
        \begin{tabular}{cc}
            \rule{6cm}{0.1pt}  & \rule{6cm}{0.1pt} \\
            Ing. Andrés Mora Zúñiga & Bach. Ana María Gomez Granados \\
            Profesor lector         & Profesor lector \\
        \end{tabular}
    \end{table}
\end{center}

\restoregeometry

% ---- Resument ----
\chapter{Resumen}

El análisis de la marcha es un estudio clínico sencillo que colabora en el diagnóstico y control de pacientes con alguna dificultad motora. El Laboratorio de Investigación en Reconocimiento de Patrones y plataformas Inteligentes (PRIS-Lab) de la Escuela de Ingeniería Eléctrica de la Universidad de Costa Rica cuenta con equipo de captura de movimiento apropiado para el estudio del movimiento humano. Desde este Laboratorio se propone la creación de una plataforma computacional que permita a profesionales de diversas disciplinas colaborar en la investigación del movimiento humano, desempeño deportivo y salud. Este trabajo representa un primer acercamiento a esta plataforma computacional, a través del desarrollo de infraestructura de software para estudio de la marcha, tema que ya se ha tratado en el laboratorio. A través de una búsqueda bibliográfica se determinaron las variables cinemáticas y espacio-temporales, así como las técnicas de análisis usadas con mayor frecuencia al estudiar la marcha. A partir de lo encontrado se desarrolló una infraestructura que permitiría a los investigadores comprobar hipótesis rápidamente y desarrollar nuevas técnicas de análisis y descubrir relaciones entre variables cinemáticas y espacio-temporales con las afectaciones motoras. 


% ---- Índices ----
\tableofcontents
\listoffigures
\listoftables

% ---- Cuerpo del trabajo ----
\mainmatter
\pagestyle{fancy}

\newchap{introduccion}

\newchap{generalidades}

\newchap{variables-tecnicas}

\newchap{solucion_propuesta}

\newchap{eval_solucion}

\newchap{conclusiones}

% ---- apéndices ----
\appendix 

\newchap{apendices}

% ---- Bibliografía ----
\bibliography{referencias}

\end{document}
