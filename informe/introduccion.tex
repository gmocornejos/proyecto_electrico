\chapter{Introducción}

Discutir sobre el problema, justificación y alcance. \\
{\Large\textbf{Voy a dejar esto para el final}}\\
Según \cite{perry} el análisis de la marcha es un estudio sencillo de interés clínico, con el cual se pueden diagnosticar distintas afectaciones, al identificar patrones en el movimiento. Algunos padecimientos que producen patrones característicos en el movimiento son: poliomielitis, amputaciones, daños del cordón espinal, traumatismo cerebral, parálisis cerebral, atrofia muscular, esclerosis múltiple, artritis reumatoide, entre otros. 

Hasta hace una década, la identificación de dichos patrones característicos del movimiento era realizada a ojo desnudo, por expertos clínicos entrenados en el área. Actualmente se pretende complementar el análisis realizado por el experto con mediciones precisas, realizadas por algún sistema de captura de movimiento: vídeo, marcadores pasivos, marcadores activos, acelerómetros, plataformas de fuerza, principalmente. Esta tendencia a incluir sensores en los análisis se analizarán en los próximos capítulos, pero en general es importante destacar la necesidad de software especializado para el análisis de la marcha. 


\section{Objetivos}

\subsection{Objetivo general}

Desarrollar un \emph{software} para facilitar a investigadores de distintas disciplinas analizar las variables cinemáticas y espacio-temporales de la marcha, a partir de datos recolectados por un sistema de captura óptica de movimiento.

\subsection{Objetivos específicos}

\begin{enumerate}
    \item Realizar una investigación bibliográfica sobre las variables cinemáticas y espacio temporales de la marcha y técnicas de análisis de la marcha.
    \item Seleccionar las variables cinemáticas y espacio-temporales más relevantes, así como las técnicas más utilizadas al realizar un análisis de la marcha. 
    \item Plantear los requerimientos técnicos y de uno para la solución. 
    \item Implementar un \emph{software} capaz de tomar datos de un sistema de captura óptica de movimiento y aplicar técnicas de análisis usuales a variables cinemáticas y espacio-temporales de la marcha, según los requerimientos técnicos y de uso planteados.
    \item Evaluar el \emph{software}, tanto a nivel técnico como de uso.
    \item Elaborar documentación del \emph{software} e informe final del proyecto.
\end{enumerate}

\section{Desarrollo}

¿Qué se va a tratar en cada capítulo? 
