\chapter[Variables cinemáticas]{Variables cinemáticas y espacio-temporales}

Según la mecánica clásica, el movimiento de un cuerpo se puede estudiar desde dos puntos de vista: la \emph{cinemática} describe el movimiento del cuerpo (posición, velocidad, aceleración) y la \emph{dinámica} estudia la fuerza y las causas que provocan que el objeto se mueva. \citep{giancoli} 

El presente trabajo se centra en la cinemática de la marcha y para su estudio se modelará el cuerpo humano como un \emph{cuerpo articulado}: un conjunto de segmentos rígidos con uniones de 2 grados de libertad. En la práctica, se suele registrar las variables cinemáticas de las uniones del cuerpo, por ejemplo \cite{menz} reporta la aceleración de la cabeza y la pelvis para estudiar la estabilidad de la marcha. Al utilizar un sistema de captura óptica del movimiento, como el mostrado en la figura \ref{fig:pris-mocap} se puede conocer las variables cinemáticas de todas las uniones del cuerpo articulado modelado por el sistema.

Debido a que las diferencias de altura (principalmente) y contextura entre personas, resulta muy difícil comparar las variables cinemáticas de manera directa, por ejemplo, la distancia y velocidad del tobillo son mayores según la altura de la persona. Para poder realizar comparaciones de los patrones de la marcha entre personas, de sin importar sus características físicas, se suele utilizar \emph{descriptores} invariables a algunas de estas características. Se ha encontrado, por ejemplo, que el ángulo que forma la rodilla (entre la tibia y el fémur) es semejante en todas las personas, sin importar su altura. 

Igualmente, existen variables \emph{espacio-temporales} independientes a las características físicas, por ejemplo, la cadencia, definida como cantidad de pasos por unidad de tiempo. Estas variables permiten comparar aspectos globales de la marcha, relacionados con el desplazamiento del cuerpo como un todo a través del espacio. 

El objetivo de este capítulo es determinar la variables cinemáticas y espacio-temporales más utilizadas al estudiar la marcha, con el fin de incluir por defecto en la solución, métodos automáticos de extracción de estas variables. Con este fin se examinarán artículos de investigación actuales y se identificarán dichas variables.










